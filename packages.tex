%%%%% General %%%%%
\usepackage[utf8]{inputenc}
\usepackage{hyperref}
\usepackage{paralist}
\usepackage{enumitem}
\usepackage{comment}

\usepackage{marginnote}
\renewcommand*{\marginfont}{\normalfont\footnotesize}

\newcommand*{\lecture}[5]{%
  \stepcounter{section}
  \section*{\thesection\quad #1%
  \marginnote{%
  \begin{tikzpicture}[framed,inner frame sep=0pt,text width=1cm,align=center]
    \node [fill=lightgray] at (0,0)                                (dayofweek)  {#2};
    \node [fill=white    ] at ($(dayofweek)  +(0,-\baselineskip)$) (dayofmonth) {#3};
    \node [fill=white    ] at ($(dayofmonth) +(0,-\baselineskip)$) (month)      {#4};
    \node [fill=lightgray] at ($(month)      +(0,-\baselineskip)$) (dayofmonth) {#5};
  \end{tikzpicture}}}
  \addcontentsline{toc}{section}{\thesection\quad #1}
  % 
}

%%%%% Graphics %%%%%
\usepackage{graphicx}

\usepackage{tikz}
\usetikzlibrary{calc}
\usetikzlibrary{backgrounds}
\usetikzlibrary{arrows}
\usetikzlibrary{positioning}

\usepackage{color}

%%%%% Math %%%%%
\usepackage{amsmath}
\usepackage{amssymb}
\usepackage{amsthm}
\usepackage{thmtools}
\usepackage{thm-restate}
\usepackage{mathtools}
\usepackage{oubraces}
\usepackage{stmaryrd}

\declaretheorem[style=plain]{theorem}
\declaretheorem[style=plain,sibling=theorem]{lemma}
\declaretheorem[style=plain,sibling=theorem]{proposition}
\declaretheorem[style=definition,sibling=theorem,qed=$\lhd$]{definition}
\declaretheorem[style=definition,sibling=theorem]{example}
\declaretheorem[style=remark,sibling=theorem]{remark}

\usepackage{proof}

%%%%% Algorithms %%%%%
\usepackage[ruled,linesnumbered,vlined]{algorithm2e}
\DontPrintSemicolon
\let\oldnl\nl% Store \nl in \oldnl
\newcommand{\nonl}{\renewcommand{\nl}{\let\nl\oldnl}}% Remove line number for one line

%%% Local Variables: 
%%% mode: latex
%%% TeX-master: "main"
%%% End: 
